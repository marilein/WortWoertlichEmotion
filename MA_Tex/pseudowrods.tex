PSEUDO	PSEUDOKAN	PSEUDOIPA	ARMIPA	ARM
tápik	ta:pIk	\textipa{[tʰɑpʰikʰ]}	\textipa{[tʰɑpʰɛl] [mɑɾtʰikʰ]}	
misuk	mizUk	\textipa{[mizukʰ]}	\textipa{[mijin] [bɑzuk] [mɔɾukʰ]}	
sawet 	zav@t	\textipa{[zɑvɛtʰ]}	\textipa{[zɑtik] [gɛtʰ]	}
nemmok	nemOk	\textipa{[nɛmɔkʰ]}	\textipa{[nɛmɾutʰ] [kɑnɛm] [kuzɛs]}	


{
	\renewcommand{\arraystretch}{1.2}
	\begin{table}[!t]
		\centering
		
		\caption{\textbf{Die folgenden Pseudowörter wurden als Stimuli ausgewählt.} Nach der Prüfung der Legalität für armenische Höörer, sowie der Qualitätsprüfung sind nur 4 Pseudowörter zur Auswahl geblieben.} \label{table:stimuli}
		
		\begin{tabular}{@{}lllll}
			
			\toprule
			
			\textbf{Stimuli} \\
			
			\midrule
			
			PSEUDO & PSEUDO KAN & PSEUDO IPA & ARM IPA & ARM\\
			tápik & ta:pIk & \textipa{[tʰɑpʰikʰ]} & \textipa{[tʰɑpʰikʰ]} & \textipa{[tʰɑpʰɛl] [mɑɾtʰikʰ]}\\
		
			
			\midrule
			
			\textbf{Control condition} \\
			
			\midrule
			
			Die Sonne lacht. & The sun is laughing.\\
			Gib mir bitte die Butter! & Please pass me the butter!\\

			
			\bottomrule
			
		\end{tabular}
		
	\end{table}
}





misuk & mizUk &	\textipa{[mizuk\super{h}]} & \textipa{[mitum] [bAzuk] [mORuk\super{h}]} & {\artm mijaket, bazuk, moruq } \\			
sawet &	zav@t & \textipa{[zAvEt\super{h}]} & \textipa{[zAtik] [gEt\super{h}]} & {\artm Zaven, havet, geth } \\
nemmok & nemOk & \textipa{[nEmOk\super{h}]} & \textipa{[nEmRut\super{h}] [kAnEm] [kuzEs]} & {\artm Nemrut, kanem, kuzes, kamoq} \\	